% Do not change document class, margins, fonts, etc.
\documentclass[a4paper,oneside,bibliography=totoc]{scrbook}

% some useful packages (add more as needed)
\usepackage{scrhack}
\usepackage[utf8]{inputenc}
\usepackage{graphicx}
\usepackage{latexsym}
\usepackage{amsmath}
\usepackage{amssymb}
\usepackage{tabularx}
\usepackage{booktabs}
\usepackage{algorithm}
\counterwithin{algorithm}{chapter}
\usepackage{algorithmic}
\usepackage{csquotes}
\renewcommand{\algorithmiccomment}[1]{\hfill\textit{// #1}}
\usepackage[usenames,dvipsnames]{xcolor}
\usepackage[colorlinks,citecolor=Green]{hyperref}
\usepackage{lipsum}

% chicago citation style
\usepackage{natbib}
\bibliographystyle{chicagoa}
\setcitestyle{authoryear,round,semicolon,aysep={},yysep={,}} \let\cite\citep

% example enviroments (add more as needed)
\newtheorem{definition}{Definition} \newtheorem{proposition}{Proposition}

\begin{document}

\frontmatter
\subject{Master Thesis}
\title{Title of Your Master Thesis}
\author{Max Lautenbach\\
  (matriculation number XXXXXXXX)}
\date{\today}
\publishers{{\small Submitted to}\\
  Data and Web Science Group\\
  Prof.\ Dr.\ Right-Name-Here\\
  University of Mannheim\\}
\maketitle

\chapter{Abstract}

% Your abstract goes here

\begingroup%
\hypersetup{hidelinks}%
\tableofcontents%
\endgroup

\mainmatter

\chapter{Introduction}
\label{ch:intro}

% Your introduction goes here

\chapter{Background}
\label{ch:related_work}
\section{Knowledge Graphs}
\begin{itemize}
  \item What are Knowledge Graphs
  \item How are Knowledge Graphs Designed (OWL, Restrictions - Domain \& Ranges)?
  \item How are Knowledge Graphs Stored (RDF, Vector Store - Fuseki)
  \item How are Knowledge Graphs Queried (SPARQL)
  \item What is the Use Case of Knowledge Graphs
  \item Common Public Knowledge Graphs (Wikidata)
\end{itemize}
\section{Closed Information Extraction}
\begin{itemize}
  \item What is Information Extraction
  \item What is Closed Information Extraction
  \item What types of approaches exist
\end{itemize}
\section{AI Agents and Tools}
\begin{itemize}
  \item What is an Agent
  \item What is an AI Agent
  \item How are Tools incorporated
  \item Which Frameworks do exist
\end{itemize}
\section{Multi-Agent-Systems}
\begin{itemize}
  \item What are Multi-Agent Systems
  \item What Architectures exist for MAS
\end{itemize}
\section{Related Work}
\begin{itemize}
  \item GenIE \cite{Josifoski2021}
  \item synthIE \cite{Josifoski2023}
  \item REBEL \cite{HuguetCabot2021}
  \item DISCIE \cite{Moeller2024}
  \item AgentRE \cite{Shi2024}
\end{itemize}

% Your literature review goes here

\chapter{Methodology}
\section{Agent Design}
\begin{itemize}
  \item Building effective agents \cite{Anthropic2024}
  \item Prompt Engineering \cite{Schulhoff2025}
\end{itemize}
\section{Datasets}
\subsection{synthIE}
\subsection{REBEL}
\section{Evaluation Metrics}
\begin{itemize}
  \item Definition Positive/Negative Triple \cite{Josifoski2021}
  \item F1-Score, Precision, Recall
  \item With Parents, With Related
\end{itemize}

% Your first main chapter goes here

\chapter{Evaluation}
\section{Evaluation Setup}
\begin{itemize}
  \item Langgraph
  \item 4 Datasplits on synthIE (Train, Train\_Text, Test, Test\_Text)
  \item Results on 5/50 Examples
  \item Cerebras, SambaNova Cloud, vLLM on 2x A100
  \item Models: Llama 3.3 70B (Mainly), Tests with o3-mini, Llama 4 ...
\end{itemize}
\section{Agent Architectures}
\subsection{Baseline}
\subsection{Supervisor}
\subsection{ReAct}
\subsection{Network}

\section{Agent Tools}
\subsection{URI Search}
\subsection{Network Traversal}
\subsection{Message Deletion}
\subsection{Semantic Validation}
\subsection{Turtle to Label}

\section{Results (A)}
\subsection{Baseline}
\subsection{Supervisor}
\subsection{ReAct}
\subsection{Network}

\section{Results (B)}
\subsection{Initial Baseline Setup}
\begin{itemize}
  \item Supervisor Agent
  \item Entity Extractor
  \item Relation Extractor
  \item URI Retrieval Agent
\end{itemize}

\subsection{Step 1: Modularization of Agent Tasks}
\begin{itemize}
  \item Decomposition of the Supervisor Agent into:
        \begin{itemize}
          \item Planner
          \item Agent Instructor
          \item Result Checker
          \item Result Formatter
        \end{itemize}
  \item Optional: Predicate Extractor
\end{itemize}

\subsection{Step 2: Handling Errors and Improving Robustness}
\begin{itemize}
  \item Integration of error messages
  \item Improved state monitoring and feedback
\end{itemize}

\subsection{Step 3: Task Simplification and State Refinement}
\begin{itemize}
  \item Simplification of agent state representations (e.g., Gen1 → Gen1v2)
\end{itemize}

\subsection{Step 4: One-Agent Architecture with Tool Usage}
\begin{itemize}
  \item Centralization of all tasks into a single agent
  \item Use of external tools to perform sub-tasks
\end{itemize}

\subsection{Step 5: Knowledge Graph Integration}
\begin{itemize}
  \item Incorporation of domain-specific knowledge via:
        \begin{itemize}
          \item Network Traversal
          \item Semantic Validation
          \item Turtle to Label conversion
        \end{itemize}
\end{itemize}

\subsection{Step 6: Gen2 Network Agent – Final Architecture}
\begin{itemize}
  \item Combination of insights from Baseline, Gen1, and One-Agent approaches
  \item Network-based collaboration between specialized agents
  \item Dynamic use of tools and agent interactions
\end{itemize}

\subsection{Step 7: Model Variation}
\begin{itemize}
  \item Various models at various stages and show results
  \item Elaborate on why to choose Llama 3.3 70B over other models
\end{itemize}

\section{Discussion}


\chapter{Conclusion and Outlook}

% Your conclusions go here

\bibliography{references/references}

\appendix
\chapter{Additional Material}

% Your additional material goes here

\backmatter
\chapter{Ehrenwörtliche Erklärung}

Ich versichere, dass ich die beiliegende Bachelor-, Master-, Seminar-, oder
Projektarbeit ohne Hilfe Dritter und ohne Benutzung anderer als der angegebenen
Quellen und in der untenstehenden Tabelle angegebenen Hilfsmittel angefertigt
und die den benutzten Quellen wörtlich oder inhaltlich entnommenen Stellen als
solche kenntlich gemacht habe. Diese Arbeit hat in gleicher oder ähnlicher Form
noch keiner Prüfungsbehörde vorgelegen. Ich bin mir bewusst, dass eine falsche
Erklärung rechtliche Folgen haben wird.

\begin{center}
  \textbf{Declaration of Used AI Tools} \\[.3em]
  \begin{tabularx}{\textwidth}{lXlc}
    \toprule
    Tool & Purpose & Where? & Useful? \\
    \midrule
    % Add your AI tools here
    \bottomrule
  \end{tabularx}
\end{center}

\vspace{2cm}
\noindent Unterschrift\\
\noindent Mannheim, den XX.~XXXX 2024 \hfill

\end{document}
